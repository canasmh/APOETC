%% Generated by Sphinx.
\def\sphinxdocclass{report}
\documentclass[letterpaper,10pt,english]{sphinxmanual}
\ifdefined\pdfpxdimen
   \let\sphinxpxdimen\pdfpxdimen\else\newdimen\sphinxpxdimen
\fi \sphinxpxdimen=.75bp\relax

\PassOptionsToPackage{warn}{textcomp}
\usepackage[utf8]{inputenc}
\ifdefined\DeclareUnicodeCharacter
% support both utf8 and utf8x syntaxes
  \ifdefined\DeclareUnicodeCharacterAsOptional
    \def\sphinxDUC#1{\DeclareUnicodeCharacter{"#1}}
  \else
    \let\sphinxDUC\DeclareUnicodeCharacter
  \fi
  \sphinxDUC{00A0}{\nobreakspace}
  \sphinxDUC{2500}{\sphinxunichar{2500}}
  \sphinxDUC{2502}{\sphinxunichar{2502}}
  \sphinxDUC{2514}{\sphinxunichar{2514}}
  \sphinxDUC{251C}{\sphinxunichar{251C}}
  \sphinxDUC{2572}{\textbackslash}
\fi
\usepackage{cmap}
\usepackage[T1]{fontenc}
\usepackage{amsmath,amssymb,amstext}
\usepackage{babel}



\usepackage{times}
\expandafter\ifx\csname T@LGR\endcsname\relax
\else
% LGR was declared as font encoding
  \substitutefont{LGR}{\rmdefault}{cmr}
  \substitutefont{LGR}{\sfdefault}{cmss}
  \substitutefont{LGR}{\ttdefault}{cmtt}
\fi
\expandafter\ifx\csname T@X2\endcsname\relax
  \expandafter\ifx\csname T@T2A\endcsname\relax
  \else
  % T2A was declared as font encoding
    \substitutefont{T2A}{\rmdefault}{cmr}
    \substitutefont{T2A}{\sfdefault}{cmss}
    \substitutefont{T2A}{\ttdefault}{cmtt}
  \fi
\else
% X2 was declared as font encoding
  \substitutefont{X2}{\rmdefault}{cmr}
  \substitutefont{X2}{\sfdefault}{cmss}
  \substitutefont{X2}{\ttdefault}{cmtt}
\fi


\usepackage[Bjarne]{fncychap}
\usepackage{sphinx}

\fvset{fontsize=\small}
\usepackage{geometry}

% Include hyperref last.
\usepackage{hyperref}
% Fix anchor placement for figures with captions.
\usepackage{hypcap}% it must be loaded after hyperref.
% Set up styles of URL: it should be placed after hyperref.
\urlstyle{same}
\addto\captionsenglish{\renewcommand{\contentsname}{Contents:}}

\usepackage{sphinxmessages}
\setcounter{tocdepth}{1}



\title{APOETC}
\date{Nov 14, 2019}
\release{0.0.1}
\author{Manuel H.\@{} Canas, Alexander Stone-Martinez, Rogelio Ochoa, Bryson Stemock, Hasan Rahman}
\newcommand{\sphinxlogo}{\vbox{}}
\renewcommand{\releasename}{Release}
\makeindex
\begin{document}

\pagestyle{empty}
\sphinxmaketitle
\pagestyle{plain}
\sphinxtableofcontents
\pagestyle{normal}
\phantomsection\label{\detokenize{index::doc}}


Welcome to APOETC’s documentation (Apache Point Observatory Exposure Time Calculator).


\chapter{About APOETC}
\label{\detokenize{index:about-apoetc}}
This exposure time calculator is specifically designed for the Astrophysical Research Consortium (ARC) 3.5m telescope. It was motivated by an observational techniques class from the Astronomy Department at New Mexico State University. The program was entirely written by graduate students so that astronomers can easily determine how long they must expose in order to obtain their desired signal to noise ratio when observing with the ARC.


\section{Documentation}
\label{\detokenize{modules:documentation}}\label{\detokenize{modules::doc}}

\subsection{The \sphinxstyleliteralintitle{\sphinxupquote{arc}} module}
\label{\detokenize{modules:the-arc-module}}

\subsubsection{\sphinxstyleliteralintitle{\sphinxupquote{Instrument}}}
\label{\detokenize{modules:instrument}}\index{Instrument (class in arc)@\spxentry{Instrument}\spxextra{class in arc}}

\begin{fulllineitems}
\phantomsection\label{\detokenize{modules:arc.Instrument}}\pysiglinewithargsret{\sphinxbfcode{\sphinxupquote{class }}\sphinxcode{\sphinxupquote{arc.}}\sphinxbfcode{\sphinxupquote{Instrument}}}{\emph{inst\_name}}{}
Bases: \sphinxcode{\sphinxupquote{object}}

This object represents the instrument used.
\begin{quote}\begin{description}
\item[{Parameters}] \leavevmode
\sphinxstyleliteralstrong{\sphinxupquote{inst\_name}} (\sphinxstyleliteralemphasis{\sphinxupquote{str}}) \textendash{} This is the name of the instrument used.

\end{description}\end{quote}
\index{filter() (arc.Instrument method)@\spxentry{filter()}\spxextra{arc.Instrument method}}

\begin{fulllineitems}
\phantomsection\label{\detokenize{modules:arc.Instrument.filter}}\pysiglinewithargsret{\sphinxbfcode{\sphinxupquote{filter}}}{\emph{bandpass}, \emph{Johnson=True}, \emph{SDSS=False}}{}
Method that returns the transmission of specified filter.
\begin{quote}\begin{description}
\item[{Parameters}] \leavevmode\begin{itemize}
\item {} 
\sphinxstyleliteralstrong{\sphinxupquote{bandpass}} (\sphinxstyleliteralemphasis{\sphinxupquote{str}}) \textendash{} The bandpass of the filter used (i.e., ‘U’,’B’,’V’,’R’, or ‘I’).

\item {} 
\sphinxstyleliteralstrong{\sphinxupquote{Johnson}} (\sphinxstyleliteralemphasis{\sphinxupquote{bool}}\sphinxstyleliteralemphasis{\sphinxupquote{, }}\sphinxstyleliteralemphasis{\sphinxupquote{optional}}) \textendash{} If true, then the bandpass is referring to the Johnson-Cousin filters. Defaults to True

\item {} 
\sphinxstyleliteralstrong{\sphinxupquote{SDSS}} (\sphinxstyleliteralemphasis{\sphinxupquote{bool}}\sphinxstyleliteralemphasis{\sphinxupquote{, }}\sphinxstyleliteralemphasis{\sphinxupquote{optional}}) \textendash{} If true, then the bandpass is referring to the Johnson-Cousin filters. Defaults to False

\end{itemize}

\end{description}\end{quote}
\begin{quote}\begin{description}
\item[{Returns}] \leavevmode
The transmission of the filter interpolated over the bandpass. Also sets a filter\_range attribute (Angstroms).

\item[{Return type}] \leavevmode
Interpolated object.

\end{description}\end{quote}

\end{fulllineitems}

\index{interpolate\_efficiency() (arc.Instrument method)@\spxentry{interpolate\_efficiency()}\spxextra{arc.Instrument method}}

\begin{fulllineitems}
\phantomsection\label{\detokenize{modules:arc.Instrument.interpolate_efficiency}}\pysiglinewithargsret{\sphinxbfcode{\sphinxupquote{interpolate\_efficiency}}}{}{}
Method that interpolates the quantum efficiency.
\begin{quote}\begin{description}
\item[{Returns}] \leavevmode
The efficiency of the instrument interpolated over the appropriate wavelenghts (in Angstroms).

\end{description}\end{quote}

\end{fulllineitems}


\end{fulllineitems}



\subsubsection{\sphinxstyleliteralintitle{\sphinxupquote{Telescope}}}
\label{\detokenize{modules:telescope}}\index{Telescope (class in arc)@\spxentry{Telescope}\spxextra{class in arc}}

\begin{fulllineitems}
\phantomsection\label{\detokenize{modules:arc.Telescope}}\pysiglinewithargsret{\sphinxbfcode{\sphinxupquote{class }}\sphinxcode{\sphinxupquote{arc.}}\sphinxbfcode{\sphinxupquote{Telescope}}}{\emph{obs\_name='ARC 3.5m'}, \emph{aperature=3.5}}{}
Bases: \sphinxcode{\sphinxupquote{object}}

Object that represents the telescope used.
\begin{quote}\begin{description}
\item[{Parameters}] \leavevmode\begin{itemize}
\item {} 
\sphinxstyleliteralstrong{\sphinxupquote{obs\_name}} (\sphinxstyleliteralemphasis{\sphinxupquote{str}}\sphinxstyleliteralemphasis{\sphinxupquote{,}}\sphinxstyleliteralemphasis{\sphinxupquote{optional}}) \textendash{} The name of the observatory used, default to ‘ARC 3.5m’.

\item {} 
\sphinxstyleliteralstrong{\sphinxupquote{aperature}} (\sphinxstyleliteralemphasis{\sphinxupquote{float}}\sphinxstyleliteralemphasis{\sphinxupquote{, }}\sphinxstyleliteralemphasis{\sphinxupquote{optional}}) \textendash{} The diameter of the telescope used (in meters), default to 3.5.

\end{itemize}

\end{description}\end{quote}

\end{fulllineitems}



\subsection{The \sphinxstyleliteralintitle{\sphinxupquote{signal\_to\_noise}} module}
\label{\detokenize{modules:the-signal-to-noise-module}}

\subsubsection{\sphinxstyleliteralintitle{\sphinxupquote{Sky}}}
\label{\detokenize{modules:sky}}\index{Sky (class in signal\_to\_noise)@\spxentry{Sky}\spxextra{class in signal\_to\_noise}}

\begin{fulllineitems}
\phantomsection\label{\detokenize{modules:signal_to_noise.Sky}}\pysiglinewithargsret{\sphinxbfcode{\sphinxupquote{class }}\sphinxcode{\sphinxupquote{signal\_to\_noise.}}\sphinxbfcode{\sphinxupquote{Sky}}}{\emph{lunar\_phase=0}, \emph{seeing=1}, \emph{airmass=1}, \emph{transmission=0.9}}{}
Bases: \sphinxcode{\sphinxupquote{object}}
\index{emission() (signal\_to\_noise.Sky method)@\spxentry{emission()}\spxextra{signal\_to\_noise.Sky method}}

\begin{fulllineitems}
\phantomsection\label{\detokenize{modules:signal_to_noise.Sky.emission}}\pysiglinewithargsret{\sphinxbfcode{\sphinxupquote{emission}}}{}{}
\end{fulllineitems}

\index{transmission() (signal\_to\_noise.Sky method)@\spxentry{transmission()}\spxextra{signal\_to\_noise.Sky method}}

\begin{fulllineitems}
\phantomsection\label{\detokenize{modules:signal_to_noise.Sky.transmission}}\pysiglinewithargsret{\sphinxbfcode{\sphinxupquote{transmission}}}{}{}
\end{fulllineitems}


\end{fulllineitems}



\subsubsection{\sphinxstyleliteralintitle{\sphinxupquote{Target}}}
\label{\detokenize{modules:target}}\index{Target (class in signal\_to\_noise)@\spxentry{Target}\spxextra{class in signal\_to\_noise}}

\begin{fulllineitems}
\phantomsection\label{\detokenize{modules:signal_to_noise.Target}}\pysiglinewithargsret{\sphinxbfcode{\sphinxupquote{class }}\sphinxcode{\sphinxupquote{signal\_to\_noise.}}\sphinxbfcode{\sphinxupquote{Target}}}{\emph{magnitude}, \emph{magsystem}, \emph{filter\_range}, \emph{SED=None}, \emph{temp=5778}}{}
Bases: \sphinxcode{\sphinxupquote{object}}

Object representing the target star.
\index{magnitude (signal\_to\_noise.Target attribute)@\spxentry{magnitude}\spxextra{signal\_to\_noise.Target attribute}}

\begin{fulllineitems}
\phantomsection\label{\detokenize{modules:signal_to_noise.Target.magnitude}}\pysigline{\sphinxbfcode{\sphinxupquote{magnitude}}}
float
The magnitude of the star you wish to observe.

\end{fulllineitems}

\index{magnitude\_system (signal\_to\_noise.Target attribute)@\spxentry{magnitude\_system}\spxextra{signal\_to\_noise.Target attribute}}

\begin{fulllineitems}
\phantomsection\label{\detokenize{modules:signal_to_noise.Target.magnitude_system}}\pysigline{\sphinxbfcode{\sphinxupquote{magnitude\_system}}}
str
The magnitude used in the above attribute.

\end{fulllineitems}

\index{filter\_range (signal\_to\_noise.Target attribute)@\spxentry{filter\_range}\spxextra{signal\_to\_noise.Target attribute}}

\begin{fulllineitems}
\phantomsection\label{\detokenize{modules:signal_to_noise.Target.filter_range}}\pysigline{\sphinxbfcode{\sphinxupquote{filter\_range}}}
tuple
The band pass of the filter (xmin,xmax).

\end{fulllineitems}

\index{SED (signal\_to\_noise.Target attribute)@\spxentry{SED}\spxextra{signal\_to\_noise.Target attribute}}

\begin{fulllineitems}
\phantomsection\label{\detokenize{modules:signal_to_noise.Target.SED}}\pysigline{\sphinxbfcode{\sphinxupquote{SED}}}
obj
If specified, this will contain the interpolated spectral energy distribution
of the target star.

\end{fulllineitems}

\index{temp (signal\_to\_noise.Target attribute)@\spxentry{temp}\spxextra{signal\_to\_noise.Target attribute}}

\begin{fulllineitems}
\phantomsection\label{\detokenize{modules:signal_to_noise.Target.temp}}\pysigline{\sphinxbfcode{\sphinxupquote{temp}}}
float
The temperature of the star. This is used only if you wish to
use Plank’s law to obtain the SED.

\end{fulllineitems}

\index{blackbody\_lambda() (signal\_to\_noise.Target method)@\spxentry{blackbody\_lambda()}\spxextra{signal\_to\_noise.Target method}}

\begin{fulllineitems}
\phantomsection\label{\detokenize{modules:signal_to_noise.Target.blackbody_lambda}}\pysiglinewithargsret{\sphinxbfcode{\sphinxupquote{blackbody\_lambda}}}{}{}
Calculates the spectrum of a blackbody from temperature temp.
\begin{quote}\begin{description}
\item[{Returns}] \leavevmode
The wavelength flux of the target as determined by a blackbody

\end{description}\end{quote}

\end{fulllineitems}

\index{convert\_to\_flux() (signal\_to\_noise.Target method)@\spxentry{convert\_to\_flux()}\spxextra{signal\_to\_noise.Target method}}

\begin{fulllineitems}
\phantomsection\label{\detokenize{modules:signal_to_noise.Target.convert_to_flux}}\pysiglinewithargsret{\sphinxbfcode{\sphinxupquote{convert\_to\_flux}}}{}{}
Convert magnitude of target star to flux.
\begin{quote}\begin{description}
\item[{Returns}] \leavevmode
The wavelength flux of the target in cgs units.

\end{description}\end{quote}

\end{fulllineitems}


\end{fulllineitems}




\renewcommand{\indexname}{Index}
\printindex
\end{document}